\documentclass[a4paper, 12pt]{article}
\usepackage[top=2cm, bottom=2cm, left=2.5cm, right=2.5cm]{geometry}
\usepackage[utf8]{inputenc}
\usepackage{amsmath, amsfonts, amssymb}
\begin{document}
	\begin{center}
\huge{História do Cálculo}
	\end{center}

	\begin{center}
Elias Sábat
	\end{center}
 
	\section{Prefácio}
Umas das maiores ferramentas da Matemática. O Cálculo Diferencial e Integral foi uma descorberta importante, tendo sido essêncial para diversos campos da ciência. Os problemas das Áreas e Tangentes datam desde de 1800 a.c com nomes indo de Arquimedes á Fermat e depois chegando a Newton e Leibnz que ,de maneira independente, foram decobertas equações funcionais.
	\section{Introdução}
O desenvolvimento do Cálculo ocorreu de ordem inversa aquela estudada nos meios acadêmicos, o Cálculo Integral surgiu antes do Cálculo Diferencial. A idea da integração teve origem em processos somatórios de certas áreas e volumes. A diferenciação surgiu mais tarde como resultado de tangentes a curvas e de questões sobre máximos e mínimos. Anos depois, descobriu-se que a integração e a diferenciação estão relacionadas, sendo uma o inverso da outra.
	\section{Cálculo Integral}
O primeiro registro que se tem de uma estimativa para o cálculo de uma superfície curva, foi escrito em aproximadamente em 1980 a.c, aonde é calculado a área da superfície de um cesto com métodos semelhante a fórmula de integração.

Outra contribuição importante e antiga para o problema da área é a questão quadratura do círculo datada 430 a.c, por Antífon, um filósofo do período socrático 

\end{document}