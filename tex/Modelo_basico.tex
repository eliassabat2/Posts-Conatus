\documentclass[a4paper, 12pt]{article}
\usepackage[top=2cm, bottom=2cm, left=2.5cm, right=2.5cm]{geometry}
\usepackage[utf8]{inputenc}
\usepackage{amsmath, amsfonts, amssymb}
\begin{document}
\begin{center}
\huge{Aula de Potências}
\end{center}

\begin{center}
Professor Elias Sabát
\end{center}
 
\section{Definição de Potência}

Seja A um número real e N um número inteiro, a expressão $A^N$ representa o produto de N fatores iguais a A. Sendo A denominado base e N denominado expoente.

$$ A^N = A \cdot A \cdot A \cdot A \cdot ... (N vezes) $$ 

\section{Propriedades de Potência}
Usamos as propriedades para facilitar a vida, bruna. vamos ver algumas mais importantes:

\subsection{Soma de Potências}
sendo A um número real e x e y números inteiros, temos que : 

$$ A^x \cdot A^y = A^{x+y}$$

\subsection{Diferença de Potências}
sendo A um número real e x e y números inteiros, temos que :

$$\dfrac{A^x}{A^y} = A^{x-y} $$

\subsection{Expoente Nulo}
Sendo A um número real, temos que :

$$ A^0 = 1 $$

\subsection{Expoente Negativo}
Sendo A um número real e x um número inteiro negativo, temos que:

$$ A^{-x}  =  \dfrac{1}{A^x}$$

\subsection{Potência de Potência}
Sendo A um número real, x e y números inteiros, temos que :

$$ (A^x)^y = A^{x \cdot y}$$
\subsection{Propriedade Fundamental}
Sendo A um número real, x e y números inteiros, temos que :

$$A^{\dfrac{x}{y}} = \sqrt[y]{A^x}$$

\section{vamos para Exercícios}
\begin{center}
Te amo, xuxu. Vê se tu estuda! bjs s2
\end{center}
\end{document}